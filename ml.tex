\documentclass[4apaper,12pt]{book}
\usepackage{amsmath}
\usepackage{index}
\usepackage[toc,page]{appendix}
\usepackage{hyperref}
\usepackage[semicolon,round,sort&compress,sectionbib]{natbib}
\usepackage{chapterbib}
\makeindex

\begin{document}

\title{AI and Machine Learning mathematics  with algorithmic implementation}
\author{mohab metwally}
\date{2020}

\maketitle
\tableofcontents

\chapter*{Introduction}
\addcontentsline{toc}{chapter}{Introduction}

\section{notation}
\begin{description}
  \item through the book i sometimes use subscript without braces, or superscript with braces, or even curly braces to indicate vector element access, or row access in the case of a matrix.
\item $(x, y) \in R^{n_x}, y \in {0,1}$ are the training data-set, where x is input,  and y is corresponding output.
\item therefore $M=\{(x^1, y^1), (x^2, y^2), ..., (x^{m}, y^{m})\}$  is the training set, and $M_{test}$ is test example.
\item X is the input of training set and $\in R^{n_x \times m}$, $X=\begin{vmatrix}X^1&X^2&...&X^{m}\end{vmatrix}$, $X^{(i)}$ is the ith column of length n, and can be written as x, or $x^{(i)}$.
\end{description}

\chapter{Logistic Regression as a neural network}

\section{definitions}

\begin{description}
\item we need an estimate function $\hat{y}$ for the input x, and weight prameters w$\in R^{n_x}, b\in R$.
\item logstic function is $\hat{y}=p(y=1|x)$, and can be defined as follows: $\hat{y}=\sigma(w^Tx+b)$, where the sigma function is defined by $\sigma(z)=\frac{1}{1+e^{-z}}$, and notice when z $\to \infty, \sigma = 1, z \to -\infty, \sigma =0$.
\item
\end{description}

\section{cost function}

\begin{description}
\item starting with a estimation linear forward model $\hat{y}$, we calculate the difference between our estimate, and the real value $y$, and through optimization we try to minimize the difference, or loss/cost through gradient descent, then we update our model's parameters.
  \item loss function is minimizing the difference between estimation ${\hat{y}, y}$, and can be defined as least squre $L(\hat{y}, y)=\frac{1}{2}(\hat{y} - y)$, but least squares leads to non-convex loss function(with multiple local minimums).
\item there are different loss functions, but the most efficient is that which maximize the difference. we can define $P(y|x^{(i)}, \theta) = h(x^{(i)},\theta)^{y^{(i)}}(1-h(x^{(i)},\theta)^{1-y^{(i)}}$, to increase the sensitivity to the training set we take the likelihood function, as the loss, $L(\theta)=\prod_{i=1}^mP(y|x^{(i)}, \theta)$ see $\mathbf{(AppendixA)}$.
\item one final step in our model is that as m get larger L tend to go to zero, to solve this we define the average sum of log-likelihood, or loss function to be our Cost function.
\item we multiply by -1 since the sum of the log-likelihood function is negative.
\item the Cost function  $J(\theta)=\frac{1}{m}\sum_{i=1}^{m}log(h(x^{(i)},\theta)^{y^{(i)}}(1-h(x^{(i)},\theta)^{1-y^{(i)}})$

\item loss function is defined as $L(\hat{y}, y)=-[ylog(\hat{y}) - (1-y)log(1-\hat{y})]$, L$\in[0-1]$.

\item cost function is defined as the average of loss function $J(w,b)=\frac{1}{m}\sum_{i=1}^{m}L(\hat{y^{(i)}}, y)$
\end{description}

\section{Gradient Descent}

\begin{description}

\item gradient descent is a way to tune the weighting parameters, the objective is the lean toward the fittest weights with respect to the least cost.
\item iterate through cost function $\mathbf{J}$ tuning with respect to weight parameters  $\mathbf{w}$, $\mathbf{b}$.
\item iterate through: $w:=w-\alpha\frac{\partial{J}}{\partial{w}}$,  $b:=b-\alpha\frac{\partial{J}}{\partial{b}}$, for tuning w, b for the least $\mathbf{J}$ possible, such that $\alpha$ is the learning rate of GD.
\item for simplicity $\partial{J}/\partial{w}$ replaced for $\partial{w}$, and similarly $\partial{J}/\partial{b}$ is replaced for $\partial{b}$.
\item forward propagation, $$\partial w = \frac{\partial{J}}{\partial{L}}\frac{\partial{L}}{\partial{\hat{y}}}\frac{\partial{\hat{y}}}{\partial{z}}\frac{\partial{z}}{\partial{w}}$$, similarly $$\partial b = \frac{\partial{J}}{\partial{L}}\frac{\partial{L}}{\partial{\hat{y}}}\frac{\partial{\hat{y}}}{\partial{z}}\frac{\partial{z}}{\partial{b}}$$.
\item $$\partial{L}/\partial{\hat{y}}=\frac{-y}{\hat{y}} + \frac{(1-y)}{1-\hat{y}}$$, $$\partial{\hat{y}}/\partial{z}=\frac{-e^{-z}}{1+e^{-z}} = \hat{y}(1-\hat{y}).$$
\item $\partial{L}/\partial{z}=\hat{y}-y$.
\item then we can deduce that the final iteration gradient descent step after calculating sigma, loss, and cost functions can be  $$w:=w-\frac{\alpha}{m}\sum_{i=1}^m\frac{\partial{L}}{\partial{b}}=\frac{\alpha}{m}X^T(\hat{y}-y)$$, and $$b:=b-\frac{\alpha}{m}\sum_{i=1}^{m}(\hat{y}-y)$$.
\end{description}

\section {Model training}
\begin {description}
\item to train a logistic regression model given data set of \textbf{{X,y}} we divide it into 20\% for testing, and 80\% for training, such that the training set is used to train our model parameters, and testing set is a separate set to test our model's predictions.

\item we call $X=\{X_{training},X_{testing}\}, y=\{y_{training},y_{testing}\}$ the data set, and
  $\{X_{training}, y_{training}\}$ the training set,and
  $\{X_{testing}, y_{testing}\}$ the testing set.
\item using $X_{training}, y_{training}$ (for the rest of the chapter, and the book i will refer to them by $X,y$ for simplicity) we start \textbf{Forward propagation} to estimate $\hat{y}$ calculate the difference between $y$, $\hat{y}$ through \textbf{Cost function} $J(y,\hat{y})$.
\item going backward to W, b we implement \textbf{Backward propagation} through $\frac{\partial{J}}{\partial{\omega}}$, and $\frac{\partial{J}}{\partial{b}}$.
\item finally we \textbf{update weight parameters} after sufficient iterations until we minimize our cost function completely.
\end {description}

\section{Forward Propagation}
\begin{description}
\item we begin by initializing our weight, and bias parameters
  $\omega$, b randomly.
  \subsection{Activation Functions}
  \item what is activation function?
\item sigmoid:
\item relu:
\item tanh:

\item Starting with input training set \textbf{X} in our model. we estimate
  $$z=\omega^TX + b$$. then $$\hat{y}=\sigma(z)$$.

\item calculate cost function $$J(y,\hat{y})$$.
\end{description}

\section{Backward Propagation}
\begin{description}
\item after evaluating the cost function $J(y,\hat{y})$ we calculate it's derivative with respect to  $\{\omega,b\}$.

\item $$\partial{\omega}= \frac{\partial{L}}{\partial{\omega}}=
  \frac{\partial{L}}{\partial{\hat{y}}}
  \frac{\partial{\hat{y}}}{\partial{z}}
  \frac{\partial{z}}{\partial{\omega}}$$
\item $$
  \frac{\partial{L}}{\partial{\hat{y}}} = -
  (\frac{y}{\hat{y}} - \frac{(1-y)}{(1-\hat{y}}) =
  \frac{\hat{y}-y}{\hat{y}(1-\hat{y})}
  $$

\item $$e^{-z}=\frac{1}{\hat{y}} - 1=\frac{1-\hat{y}}{\hat{y}}$$
\item $$
  \frac{\partial{\hat{y}}}{\partial{z}} =
  \frac{e^{-z}}{1+e^{-z}} = (\hat{y})^2e^{-z}
  $$
\item $$ \partial{\omega} = X^T(\hat{y} - y) $$
\item $$ \partial{b} = \hat{y} - y $$
\end{description}

\section{Update parameters}
\begin{description}
  \item we implement the following algorithm with a fixed number of iteration that is customized per application, and tuned by the Engineer, such that each application would require different tuning parameters from which is the iteration number.
\item we iterate the following: update the parameters $\omega$, b, in the back propagation step using $\partial{\omega}$, $\partial{b}$.
\item $$\omega = \omega - \frac{\alpha}{m}X^T(y-\hat{y})$$
\item $$b = b - \frac{\alpha}{m}(y-\hat{y})$$

  \end{description}

\section{Summary}
\begin{description}
  \item
\end{description}

\section{Logistic Regression in Python}
\begin{description}
  \item
\end{description}

\section{References}
\begin{description}
  \item
\end{description}

\chapter{Neural Networks}
\section {Lingua franca}

\begin{itemize}
\item RELU Activation Function: It turns out that using the Tanh function in hidden layers is far more better. (Because of the zero mean of the function). Tanh activation function range is [-1,1] (Shifted version of sigmoid function). Sigmoid or Tanh function disadvantage is that if the input is too small or too high, the slope will be near zero which will cause us the gradient decent problem. RELU stands for rectified linear unit, it's a rectifier Activation function and can be defined as $f(x)=x^+=max(0,x)$ or $\begin{cases} 0 & x\leq 0 \\ x & x > 0 \end{cases}$ relu shows to be better replacement to sigmoid function $\sigma$ for the reason that it help in vanquishing gradient problem.
\item Neuron: is a linear regression algorithm denoted by z, or a, $z= W^TX + b$, such that W is the the weight vector of the network.
\item Shallow Layers: also known as Hidden Layers, is a set of neurons, for example of the network of composed of input $X$, and output $Y$, with at least a single layer $L1$, and at most 2 layers, then the forward propagation will be as follows: we calculate the logistic function for the first layer $(1)$,  $z^{1}_i=w^TX_i+b_i$, $\hat{Y}^{(1)}=\sigma{(z^{1}_i)}$ , then we proceed to calculate the final logistic evaluation for the output layer with $\hat{Y}^{(1)}$ as an input instead of $X$, and so on we proceed replacing $\hat{Y}^{(i)}$ instead of X as new input.
\item Layer: layer $L_{(i)}$ is $\hat{Y}^{(i)}=[\hat{y}^{(i)}_1, \hat{y}^{(i)}_2, ..., \hat{y}^{(i)}_n]$ such that n is the length of the layer $L_{(i)}$. each $\hat{y}^{(i)}_{j}$ is weighted with unique weight vector with previous layer $L_{(i-1)}$.

\item Neural Network(NN): is a set of interconnected layers, $<X, L_1, L_2, ..., L_m, Y>$

\item Deepness: shallow layer as defined to consist of 1-2 hidden layers, but on the other hand Deep Network is consisting of more than 2 inner, or hidden layers.

\end{itemize}

\begin{description}
\item we discussed in previous chapter that $\frac{\partial{\hat{y}}}{\partial{z}}$, is actually for the logistic activation function $\sigma$ only, we need to calculate the same derivation for tanh, and RELU.
\item for $tanh(z)=\frac{e^{z}-e^{-z}}{e^{z}+e^{-z}}$ is $\frac{\partial{\hat{y}}}{\partial{z}}=1-tanh(z)^2$
\item for relu activation function $\frac{\partial{\hat{y}}}{\partial{z}}=\begin{cases}0 & z<0\\1 & z\ge0\end{cases}$
\item in NN there are plenty of parameters to worry about, for example the weights need to be initialized randomly, with small values, and $b$ can be initialized as zero.
\end {description}

\section{Model training}
\begin{description}
\item Training a deep neural network is analogous to training a logistic regression network, as discussed in previous chapter, a logistic regression model can be considered a neural network with zero \textbf{hidden layers}.
\item We follow the same algorithm, parameter initialization, but in this case we initialize the parameters for each layer, assume a network composed of 2 hidden layers $X \rightarrow L1 \rightarrow L2 \rightarrow Y$, layer $(L^{(i)}, L^{(i-1)})$ are interconnected with weight,bias parameters $\omega^{(1)},b^{(1)}$, such that $\omega^{(i)}$ is a matrix of shape $(length(L^{(i)}), length(L^{(i-1)})$, and $b$ is a vector of shape $(length(L^{(i)}, 1))$, so each node in the layer $L^{(i)}$ is connected with each node in previous layer $L^{(i-1)}$.
\item W, b are ought to be randomly initialized, in the logistic regression discussed in previous chapter, W, b should be initialized with zero values, but in the case of the neural network zero initialization leads to $\hat{y}=0$, and all the nodes share the same weight which violates the purpose of the nodes in the first place which is to capture features from the data set as much as possible, so random initialization is necessary, and not to overshoot, initialization better in range $]0-1[$ weighted by small value around 0.01(this will be discussed in details in NN hyperparameters chapter) to reduce the sensitivity, in order for the gradient descent to not take for ever.
  \item forward and back propagation are executed in a similar way to logistic regression with minor differences.
  \item calculation of forward propagation using inputs from the previous layer instead of input data set, activation function can be \textbf{sigmoid}, \textbf{relu}, or \textbf{tanh}, it has been tested that \textbf{relu} activation function in the first layers shows better results, and \textbf{sigmoid} activation in the last layer to fit perfectly with the output classification y-vector in the range $[0-1]$

\end{description}

\section{Parameter initialization}
\begin{description}
\item Iterating through each layer, such that $L^i$ layer of length $l_i$ nodes, and previous layer $L^{i-1}$ of $l_{i-1}$ nodes, the weight parameter at layer L is inter-connected with all nodes in the previous layer, meaning that nodes at layer $L^i$ ought to have weight matrix of shape $(l_i, l_{i-1})$, and bias of shape $(l_i,1)$.
\item Unlike the logistic regression, in Neural network initialization step is crucial step, in logistic regression there is only a single activation node extracting a single feature such as price of a house for example, but we intend to employ neural networks to capture as many features as possible inside each node, and therefor initialization ought to be with random values, otherwise we end up with similar copies of each node forward, and backward propagation.
\item and since we choose random $W^i$ we can choose $b^i$ to be zero.
\end{description}

\section{Forward Propagation}
\begin{description}
\item similar to logistic regression forward propagation, done for each layer, with little difference that instead of using sigmoid function $\sigma$ we replace it with \textbf{relu} function, which is shows better results, and activate the last layer with sigmoid function to distribute the output toward the extremes.
\item the forward propagation step is done on the input $A^{i-1}$ from previous layer, with current layer $L^i$ parameters $\omega^i$, $b^i$.
  \item we iterate through layers $L^i$ such that $i \in [0,L]$
\item  $$z^i=(\omega^i)^TA^{(i-1)}+b^i$$.
\item $$
  A^i=\begin{cases}
  \text{relu{($z^i$)} $\leftarrow$ if i $<$ L} \\
  \text{$\sigma(z^i)$ $\leftarrow$ if i = L}
  \end{cases}
  $$

\end{description}

\section{Backward Propagation}
\begin{description}
\item $$
  \frac{\partial{L^i}}{\partial{\omega^i}}=
    \frac{\partial{L^i}}{\partial{A^{(i-1)}}}
      \frac{\partial{A^{(i-1)}}}{\partial{z^i}}
        \frac{\partial{z^i}}{\omega^i}
        $$
      \item $$\partial{A^{(i-1)}}=\partial{z^i}
        (\frac{\partial{A^{(i-1)}}}{\partial{z^i}})^{-1}=
        \partial{z^i}\frac{\partial{z^i}}{\partial{A^{(i-1)}}}
        $$

      \item since the last activation function is sigmoid, then $$\partial{z^L}=y^L-\hat{y^L}$$
        \item the back propagation algorithm start with the following initialization step:.
        \item $$ \partial{A^{(L)}} = \frac{(\hat{y}-y)}{\hat{y}(1-\hat{y})} $$
        \item $$\partial{A^{(L-1)}} = (\omega^{(L)})^T(y^{(L)}-\hat{y^{(L)}})$$
        \item $$ \frac{\partial{z^i}}{\partial{A^{(i-1)}}} = \omega^T $$
        \item $$\partial{A^{(i-1)}}=\omega^T\partial{z^i}$$

\end{description}

\section{Summary}
\begin{description}
\item
\end{description}

\section{Deep neural networks in Python}
\begin{description}
\item
\end{description}

\chapter {Neural Networks hyperparameters}
\begin{description}
\item machine learning have a wide applications in ranging from Computer Vision, Natural Language Processing, Speech recognition, and structured data, in data-science, and the the parameterization varies greatly from one domain to another, it's domain specific.
  \subsection{training}
\item  given a data set, wide range of models, the data  data is divided into three categories, the \textbf{training}, \textbf{development}, and the \textbf{testing}.
\item we train different models to evaluate the most fit model on the development(also called evaluation set), after choosing our model we test on the testing set.
\item the partitioning ratio of the data-set varies with the size of the data set, for example in a small data set of size 1000, it's convenient to divide the data into 70\%, 30\% for training, and test respectively, or 60\%, 20\%, 20\% for training, evaluating, and testing respectively, but in the era of big data, of millions of entries, data set is conveniently divided into (98\%, 1\%, 1\%), (99.5\%, 0.4\%, 0.1\%) for training, evaluation, testing sets respectively, the ratio varies from one domain to another.
\item for faster, and accurate training/testing the data-set ought to be for the same source of the same general range of features, on the other hand in the mismatched data-set, in the case of training set from specific source, and development/test set from different set, the development/testing process can be inefficient, so the division of the partitioned sets need to be taken randomly from the same source.
  \subsection{bias-variance trade-off}
\item In an n-dimensional training model, the bias-variance dimension indicates how far our model captures the data, and can be classified into under-fitting high-bias model where it hardly fit the data, on the other extreme, high-variance, over-fitting model, and in between just-right model.
\item high-variance: in a classification problem where the training set error is 1\%, while the dev set error is 11\%, this gape indicate that the model is well trained in a very narrow data-set with limited range of features quite different from that of the development set, in this case the model is over-fitting, or hardly scaling to other data-sets, and therefore fails in application.
\item high-bias: similarly in the same experiment if the training set error is around 15\% where the dev-set error is ~ 16\%, indicating that the model generalize much better than the previous case of high-variance over-fitting model, but with high error would indicate that the model is highly-biased, and well generalized.
\item high-bias, high-variance: the worst model is that which do not generalize well, and doesn't fit our data, with high training error say 15\%, and 30\% development error.
\item low-bias, low-variance: the perfect model is that of low bias, low-variance, that which fits the training set, and generalizes well.
  \subsection{recipes for high-bias, high-variance}
\item high-bias solution: getting a high bias means that our model doesn't capture sufficient features, and this can be due to  our network is short, or narrow, or the gradient descend need to be run for longer time, or through using optimization, or change the neural network architecture.
\item high-variance solution: high variance indicate the lack of generalization, and is reduced through training on wider data-set, regularization, and changing the NN-architecture.

  \subsection {Regularization (weight decay)}
\item in the case of high-variance, we need our model be more generalized over the data-set, one way to do so is through regularization, or changing the sensitivity of the weight parameters, this sensitivity is measured in the back-propagation step through $\partial{J}/\partial{w}$, we can vary the sensitivity by regularization parameter $\lambda$ added to the cost function  as follows:
  $$ J(w^1,w^2...,w^L,b^L,b^L...,b^L) = \frac{\alpha}{m}\sum_{i=1}^{m}{L(w^1,w^2...,w^L,b^1,b^2...,b^L)} + \frac{\lambda}{2m} \sum_{l=1}^{L} {\left\Vert w^{i} \right\Vert^2}  + \frac{\lambda}{2m}b^L $$

\item but usually the last term is quite ineffective in regularization, and ignored, so it's cuts down to the following:

\item $$ J(w^1,w^2...,w^L,b^L,b^L...,b^L) = \frac{\alpha}{m}\sum_{i=1}^{m}{L(w^1,w^2...,w^L,b^1,b^2...,b^L)}  + \frac{\lambda}{2m}  \sum_{l=1}^{L} {\left\Vert w^{i} \right\Vert^2} $$

\item where the last norm is frobenius norm:
  $$ \sum_{i=1}^{n^{(l)}} \sum_{j=1}^{n^{(l-1)}} (w_{i,j}^{(l)})^2 $$

\item backward propagation step:

\item  $$ \frac{\partial{J}}{w^{i}} = \frac{1}{m}\sum_{i=1}^{m}{\frac{\partial{L}}{\partial{w^L}}}  + \frac{\lambda}{m}  \sum_{l=1}^{L} {\left\Vert w^{i} \right\Vert} $$

\item looking at our mechanism now, $\lambda$ is a device for tuning the weight parameter, or rather as a valve to control it, the larger the $\lambda$ is the lower the w as we minimize our cost function.

  \subsection {Inverted Dropout Regularization}
\item instead of reducing the weight parameters, we can drop it out completely by assigning those values to zero.
\item implementation algorithm: by matching a threshold probability = $P_{keep}$ against activation matrix $A^{(i)}$, and to keep balanced activation values, it's re-evaluated as $A^{(i)}/(1-P_{keep})$ and this last step distinguish the inverted dropout, from normal dropout, and the inverted is more efficient.

  \subsection{Input Normalization}
\item since there are no constraint on the input features, each feature can be of different range, and the cost function can end up elongated at one dimension, and narrow on the other, being steep on one edge of the function, and smooth on the other, it can be visualized by taking a contour of the \textbf{J} function, it will look like very long ellipse.
\item with un-normalized input the gradient descend tends to take long time, and longer way down to the local minima, but if the input is normalized (demeaned, and divided by the standard deviation of the input) we end up with circular counter, and smooth, and faster gradient descent.

\subsection{Vanishing/Exploding gradients}
\item let's look deeper inside a deeper network, through L layers, such that L is large, $\hat{y}=(W^LW^{(L-1)}W^{(L-2)}...W^1)X$ if we assume $b=0$ for simplicity, and that $W^L$ is the transpose of $w^L$, in case entries of similar weight matrices in values, if the values of each $W^i$ matrix below the 1, then the overall $(W^LW^{(L-1)}W^{(L-2)}...W^1)$ term will be very small, and gradient descent will take forever, on the other hand, if those entries greater than 1, then the overall term will be huge, and the gradient descend will overshot(exploding).

\item a well chosen initialization parameters can Speed up the convergence of gradient descent, and Increase the odds of gradient descent converging to a lower training (and generalization) error.

    %TODO name the formula, and elaborate.
\item to solve vanishing/exploding gradients problem we take two steps in weight parameters initialization, first the values are choosing at random through normal distribution, secondly, setting up the variance of $w^i$ to $\frac{1}{k}$  such that k is the number of the neurons in layer $i-1$. and it's by multiplying of $w^i$ by $\sqrt{\frac{1}{k}}$, this ratio shows to fit better with relu activation if replaced with $\sqrt{\frac{2}{k}}$

\item back to inverted dropout algorithm, due to the vanishing/exploding gradients problem we fix a threshold $P_{keep}$ and compare each entry in $W_i$ (taken from uniform distribution [0-1], unlike previously we choose W at random from normal/Gaussian distribution), so.
\item back propagation step we run the same masking algorithm over $\partial{A^{l}}$ and divide it by the same factor.
\item if $A^{l}=\frac{mask(A^{l})}{P_{keep}}=\frac{A^{l}.*A_{mask}}{P_{keep}}$, then $\partial{A^{l}}=\frac{\partial{A^{l}}.*A_{mask}}{P_{keep}}$ note that $.*$ means element-wise multiplication.
  \subsection {gradient checking}
  \item we verify the GD isn't converging.
\item it's a method to verify that the gradient descent is within diminishing $\epsilon$ of error difference with the manual derivative of the derived function $f(\theta)$, and $\theta$ is a vector of unwrapped weight, and bias parameters.

\item if following the Euclidean distance $L_2$ norm difference $ < (\epsilon =10e-7)$ then the result is negative. $$
  \frac{
    \left\Vert\frac{\partial{J(\theta^i)}}{\partial{\theta^i}} - \frac{\nabla {J(\theta^i)}}{\nabla{\theta^i}}\right\Vert
  }
       {
         \left\Vert\frac{\partial{J(\theta^i)}}{\partial{\theta^i}}\right\Vert +
         \left\Vert\frac{\nabla{J(\theta^i)}}{\nabla{\theta^i}}\right\Vert }  < \epsilon $$
\end{description}

\section {Optimization}

\begin {description}
\item machine learning is highly iterative process, to choose low-bias, low-variance model from the set of trained/tested models, the iteration process for each model through the same m data-set entries, of L layers, and this process is time consuming, optimized algorithms are essential.
  \subsection{stochastic, mini-batch, and batch gradient descent}
\item let's consider the input data-set $X$ of image-classification model Neural Network, using big-data of 5 million images, each with resolution of $800*400$, and three color channels, then our input Matrix X would be of size $5M*800*400*3$ bytes need to be loaded in the training machine memory, that is around 4.37 Tera bytes!
\item to be able to train our model we divide our Input set into mini-batches, if we choose the batch size to be 1 meaning we run the gradient descent on a single input entry, without taking benefit of vectorization, and machine GPU, then it's called \textbf{stochastic gradient descent} it's a \textbf{mini-batch} of size 1, even using input from the same distribution the cost function hardly reduces to the local minima, and tend to take random-walks for each input, it's hard to predict it's behavior, but for the overall training-set it converges to the neighbourhood area of the local minima. but on the bright side it can run on a moderate computer, but takes longer time compared to \text{batch gradient descent} that make full use of vectorization, and GPU power.
\item if we take the benefits of both worlds and choose \text{mini-batch} size fit enough to run in our training machine GPU/CPU memory, we can run our model in the  optimal time possible. so instead of running (X,Y) of size $m$ inside our model, we divide our training data-set (X,Y) into $(X^{\{t\}}, Y^{\{t\}}) | t=\frac{m}{number-of-batches}$ mini-batches.

\end{description}

\section{Gradient descent with momentum}
\begin{description}
\item  In the case of min-batch gradient descent, or in slow gradient descent the descent tends to overshoot, and deviate from the shortest path, to solve overshooting we employ weighted averages of the gradients (see appendix (D) on exponentially weighted averages) as follows: $$ Vdw^i = \beta_1 Vdw^{i-1} + (1-\beta_1) dw^i $$
\item then we use $Vdw^i$ in update parameters step, $w^i = w^i - \alpha Vdw^i$.
  \end{description}

\section{RMSprob}
\begin{description}
\item In a highly oscillating converging gradient descent around the axis of convergence, we can employ 'root mean squared prob' algorithm also using weighted averages, and here is the formula $$ Sdw^i = \beta_2 Sdw^{i-1} + (1-\beta_2) (dw^i)^2 $$, the only difference here is that the last term's parameter is squared, and it's useful in the case of oscillating gradient along side an axis perpendicular to the converging axis, the squared weight will boost relative large value, or shrink down relatively small value, then in the update step we divide by the square root of the calculated maximized/shrinked weight value, if it was initialize relatively small we can compensate by increasing the learning rate $\alpha$ to leap larger steps, if the initial $dw$ value was relatively large then we shrink it for more stable gradient descent.
\item the last step is: $$ w^i = w^i - \alpha \frac{dw^i}{\sqrt{Sdw^i + \epsilon}} $$.

\end{description}

\section{Adaptive Momentum estimation (Adam)}
\begin{description}
\item It's not always the case that one optimization algorithm on first application will scale will on a the second, so combination of different optimization algorithms shows to work best.
\item adam algorithm combines both RMSprob, and momentum gradient descent:
  $$ w^i = w^i - \alpha \frac{Vdw_{corrected}^i}{\sqrt{Sdw_{corrected}^i + \epsilon}} $$
  $$ b^i = b^i - \alpha \frac{Vdb_{corrected}^i}{\sqrt{Sdb_{corrected}^i + \epsilon}} $$
  (look up appendix D for further explanations on bias correction)
  \subsection {Hyperparameters}
  \begin{description}
  \item the overall number of parameters is increasing: \begin{itemize}
  \item $\alpha$: need to be tuned.
  \item $\beta_1$:set to 0.9 (can be tuned).
  \item $\beta_2$: set to 0.99 (can be tuned).
  \item $\epsilon$: set to $10^{-8}$.
  \end{itemize}
  \end{description}
\end{description}

\section{Learning rate decay}
\begin{description}
\item at the start of gradient descent, it start making large leaps away from the maximum minima, but as it converges, gradient can converge faster if the $\alpha$ is tuned to adjust itself to get smaller over time.
\item in  gradient descent iterations are called epochs, of index $epoc_{num}$ an epoch with rate $decay_{rate}$, we initialize learning rate to relatively large value $\alpha_0$ then we set our decaying learning rate to: $$\frac{1}{1+epoch_{num}*decay_{rate}}\alpha_0$$
\item there are different variation of learning rate decay of the form scale*$\alpha_0$ for example:
  $$0.95^{epoch_{rate}}\alpha_0 $$
\end{description}

\section{Tuning parameters}
\begin{description}
\item so far our highly optimized neural network is composed of many parameters left to the designer to tune (ordered by priority):
  \begin{enumerate}
\item learning rate \textbf{$\alpha$}.
\item number of hidden unites.
\item momentum averaging constant \textbf{$\beta$}.
\item mini-batch size.
\item number of layers \textbf{L}.
\item learning rate decay initial value \textbf{$\alpha_0$}.
\item ADAM's $\beta_1$, $\beta_2$, $\epsilon$.
  \end{enumerate}
\item so we have 7 parameters so far, the permutation of those 7 variables even in a narrow range of discrete values, can take for ever, and an old-practice was to pick random permutation on a grid, meaning that $hyper-parameter_i$, against $hyper-parameter_j$, such that each cell map two different parameters, but since we are solving a stochastic search problem, it's more efficient to set the values at random in both dimension so for example for each $hyper-parameter_i$  is projected on the $hyper-parameter_j$ at each point, and texted, this opens more possibilities.

 \subsection{coarse to fine search}
  \subsection{panda vs caviar training approaches}
  \begin{description}
  \item depending on the computational capacity available you choose between panda, and caviar approaches, training a single model at a time, and tuning it's parameters every set of epochs in case you have limited computational resources is called panda approach, on the other hand, if you have sufficient power to run different models with different parameters simultaneously, this last approach is called caviar (those terms were coined by Andrew Ng in a coursera specialization).
  \end{description}
  \end{description}
  \section{Batch Normalization}
  \begin{description}
\item in previous chapter we discussed input normalization, for the same reasons we can implement the same method for each layer's input before the activation process, then we re-scale by $\lambda,\beta$: $$
  z_{norm}^{[l]} = \frac{z^{[l]} - \mu({z^{[l]}})}{\sqrt{\sigma^2+\epsilon}} | l \in [1-L] $$
  $$\widetilde{z^{[l]}}  = \lambda z_{norm}^{[l]}  + \beta $$
\item but make sure here that $(\beta,\lambda)!=(\sqrt{\sigma^2+\epsilon}, \beta)$, because in such case we end up with  $\widetilde{z^{[l]}} = z_{norm}^{[l]}$. here $\lambda, \beta$ are learned parameters, and updated during the training.
\item update $\lambda, \beta$ in the update step:
  $$ \beta^{[l]} = \beta^{[l]} -\alpha d\beta^{[l]} $$
  $$ \lambda^{[l]} = \lambda^{[l]} -\alpha d\lambda^{[l]} $$
\item in case of implementation of batch normalization we end up with the following: $$ z_{norm}^{[l]} = \frac{z^{[l]} - \mu({z^{[l]}})}{\sqrt{\sigma^2+\epsilon}} | l \in [1-L] $$, and  $$\widetilde{z^{[l]}}  = \lambda z_{norm}^{[l]}  + \beta $$ if we look closely we found that $\beta$ also serves the purpose of bias, or shifting term \textbf{b} in forward algorithm: $z^{[l]} = w^TX^{\{i\}} + b$, therefore we can set \textbf{b} to zero, or remove this term.
  \subsection{Covariate Shift}
\item if our data-set was from specific distribution, if we cut through the neural network at layer l, then all inputs $a^{[l-1]}$ feeding layer l are depended on the tuned parameters, gradients, and basically our input data, hence if $a^{[l]}$ are projected in specific neighbourhood  for mini-batch $X^{\{i\}}$, at different mini-batch $X^{\{j\}}$ the projection can shift greatly in different area. Batch normalization help to muzzle this overshoots, by mapping the output of previous layer $l-1$ into normalization dimension, hence keeping the output more controlled for different mini-batches.
  \subsection{Batch normalization as a regularization technique}
\item analogous to regularization, each mini-batch is scaled by mean, and variance, with the effect of randomizing the network layers output, similar to dropout technique, therefor it can be considered as regularization technique.
  \subsection{Batch normalization on test sets}
\item during the test time, you can run the model on a a single example, so there are two approaches for calculation of $\mu, \sigma^2$ those values can be taken from the final mini-batch, or through exponential/running weighted averages run over each mini-batch parameters.
  \end{description}

  \section{Multi-class classification}
  \begin{description}
  \item so far we discussed a \textbf{sigmoid} activation over \textbf{$\hat{y}$}, what if our model's \textbf{y} is multi-class, and y isn't a single number but a vector of length n?
    \item if we run for example sigmoid, or relu on a vector rather than a real number the sum of the vector entries will be arbitrary number,
    \subsection{Softmax activation}
  \item softmax is a probabilistic activation function, run over $z^{[L]}$ is defined as: $$
    a^{[L]}=\frac{e^{z^{[l]}}}{\sum_{i=1}^{n^{[L]}}t_i},$$  where $t=e^{z^{[L]}} $.
    \item the resultant $\sum_{i=1}^{n^{[L]}}{a_i^{[L]}}=1$, where $n^{[L]}$ is the number of neurons, or nodes in the final layer, which is the same as the number of classes at output \textbf{y}.
  \end{description}



\chapter{Natural language processing}
\section{pre-processing}
\begin{description}
\item the problem here is how to extract features $\mathbf{X}$ from the a sentence.
\item for example how to classify a sentence being positive, or negative, assigning 0 for negative, and 1 for positive, starting for a preprocessed sentence how to turn it into a feature set X.
  \item but there are unnecessary punctuation, conjugation, and stops that need to be get rid of, so first we need to pre-process our data-set as follows:
  \begin{enumerate}
  \item iliminate handles and URLs
  \item tokenize the string into words
  \item remove stop words such as ``and, is, a, on, etc''
  \item covert every word into it's stem form
  \item lower case transformation
  \end{enumerate}

\end{description}
\section{Example: positive, negative classifier}
\begin{description}
\item given sentence s=``i love NLP, therefore i study it'' how to classify s=['i', 'love', 'NLP', ',', 'therefore', 'i', 'study', 'it']  as positive or negative?.
\item for a set of strings $S=\{s_1, s_2, .., s_n\}$, matching each string against a vocabulary of all words to end up with two vectors of word frequency, we create positive frequency vector freqs(w,1), and negative frequency vector freqs(w,0) such that w stand for sentence word, and such that $X_m=[1, \sum_{w}freqs(w,1), \sum_{w}freqs(w,0)]$
\item given a set of sentences, each is labeled for training as either positive, or negative. we mark each word in a positive-labeled sentence as positive, even if the word is negative, and conversely the opposite with negative-labeled sentences, we mark every word as negative.
\item first of all we create a set vocabulary $V$ that includes all sets, or sentences $S$, $V=\{words(s_i) | s_i \in{S}\}$.
\item for example in training sets ${s_1, s_2}$, $s_1$=''i love NLP, therefore i study it'' labeled as positive, and $s_2$=''society no longer in need for black magic, or superstition'' labeled negative, we can extract pos-neg features against vocabulary V=['i', 'love', 'NLP', ',', 'therefore', 'study', 'it', 'society', 'no', 'longer', 'in', 'need', 'for', 'black', 'magic', 'or', 'superstition'], pos-freqs = [2, 1, 1, 1, 1, 1, 1, 0, 0, 0, ..., 0], neg-freqs = [0, 0, 0, 1, 0, 0, 0, 1, 1, 1, 1, 1, 1, 1, 1, 1, 1]. $X_m$=[1, 8, 11]. for m training sets, $X_{m}=\begin{vmatrix}1&x_1^{(1)}&x_2^{(1)}\\1&x_1^{(2)}&x_2^{(2)}\\1&x_1^{(3)}&x_2^{(3)}\\...\\1&x_1^{(m)}&x_2^{(m)}\end{vmatrix}$
  \end{description}
\section{Logistic regression classifier}
\begin{description}
\item let's see how this fits inside the Gradient Descent algorithm, for $\sigma(X^i, \theta^i)=\frac{1}{1+e^{-z}}$, we add the bias b to the $X^i$ itself, and therefore $z=\theta^TX^i$.
\item therefore for a positive z we get  $\sigma>0.5$, and inversely for negative z we have $\sigma<0.5$.
\item now we ready for Gradient Descent, given initial parameters $\theta$, we predict, or evaluate the logistic:
\item$\sigma(X^i, \theta_j)$, then loss function $L(\hat{y}, y)=-[\hat{y}log(\hat{y}) + (1-y)log(1-\hat{y})]$, gradient $\nabla=\partial{J}/\partial{\theta_j} = \frac{X^T}{m}(\hat{y} - y)$, updating $\theta_j := \theta_j - \alpha\nabla$. iterate through gradient descent k times.
\end{description}

\section{Naive Bayes classifier}
\begin{description}
\item The priorpp probability represents the underlying probability in the target population that a tweet is positive versus negative.  In other words, if we had no specific information and blindly picked a tweet out of the population set, what is the probability that it will be positive versus that it will be negative? That is the "prior".

\item The prior is the ratio of the probabilities $\frac{P(D_{pos})}{P(D_{neg})}$.
\item We can take the log of the prior to rescale it, and we'll call this the logprior

\item $$\text{logprior} = log \left( \frac{P(D_{pos})}{P(D_{neg})} \right) = log \left( \frac{D_{pos}}{D_{neg}} \right)$$.

\item Note that $log(\frac{A}{B})$ is the same as $log(A) - log(B)$.  So the logprior can also be calculated as the difference between two logs:

\item $\text{logprior} = \log (P(D_{pos})) - \log (P(D_{neg})) = \log (D_{pos}) - \log (D_{neg})$

\item To compute the positive probability and the negative probability for a specific word in the vocabulary, we'll use the following inputs:

\item $freq_{pos}$ and $freq_{neg}$ are the frequencies of that specific word in the positive or negative class. In other words, the positive frequency of a word is the number of times the word is counted with the label of 1.

\item $N_{pos}$ and $N_{neg}$ are the total number of positive and negative words for all documents (for all tweets), respectively.
- $V$ is the number of unique words in the entire set of documents, for all classes, whether positive or negative.

\item We'll use these to compute the positive and negative probability for a specific word using this formula:

\item $$ P(W_{pos}) = \frac{freq_{pos} + 1}{N_{pos} + V}$$
\item $$ P(W_{neg}) = \frac{freq_{neg} + 1}{N_{neg} + V}$$

\item Notice that we add the "+1" in the numerator for additive smoothing.

\item To compute the loglikelihood of that very same word, we can implement the following equations:

\item $$\text{loglikelihood} = \log \left(\frac{P(W_{pos})}{P(W_{neg})} \right)$$

\item $$ p = logprior + \sum_i^N (loglikelihood_i)$$

\item Some words have more positive counts than others, and can be considered "more positive".  Likewise, some words can be considered more negative than others.
\item One way for us to define the level of positiveness or negativeness, without calculating the log likelihood, is to compare the positive to negative frequency of the word.
\item Note that we can also use the log likelihood calculations to compare relative positivity or negativity of words.
\item We can calculate the ratio of positive to negative frequencies of a word.
\item Once we're able to calculate these ratios, we can also filter a subset of words that have a minimum ratio of positivity / negativity or higher.
\item Similarly, we can also filter a subset of words that have a maximum ratio of positivity / negativity or lower (words that are at least as negative, or even more negative than a given threshold).

\item $$ \text{ratio} = \frac{pos_{words} + 1}{neg_{words} + 1} $$

\item In prevision section a Logistic regression classified, but we can quick solve the same problem in much simpler algorithm through the evaluation of the likelihood of a sentence being positive matched against given Vocabulary table $V$.
\item Recall that conditional probability $p(w_i|pos) = \frac{p(w_i\cap{pos})}{p(pos)}$, $p(pos)=freq(pos)/total$, $p(w_i\cap{pos})=freq(w_i)/total$, then $p(w_i|pos)=freq(w_i)/freq(pos)$
\item Likelihood of a positive is defined as $\prod_{i=1}^{m}{\frac{p(w_i|pos)}{p(w_i|neg)}}$, if likelihood $>$ 1 then sentence is positive, otherwise, it's negative.
\item to reduce the sensitivity of each word, and avoid getting 0 $p(w_i|class)$, $class\in \{pos, neg\}$ we modify the conditional probabilistic frequency using the so-called 'laplacian smoothing': $p(w_i|class)=\frac{freq(w_i|class)+1}{freq(class)+unique(V)}$, freq(class) is defined as $N_{class}$, and unique(V) is defined as $N_V$, for example $p(w_i|pos)=\frac{freq(w_i|pos)+1}{N_{pos}+N_V}$.
\item to keep the scale small as possible likelihood is replaced with log of likelihood coined with symbol $\lambda(w_i)=log(\frac{p(w_i|pos)}{p(w_i|neg)})$, and a prior=$log(\frac{p(pos)}{p(neg)})$, the classifier of a sentence $W$ is equivalent to prior + $\sum_{i=1}^{m}{\lambda{w_i}}$, and if $\lambda>0$ then it's positive, and negative otherwise.
\end{description}
\section {costine similaritis}
\begin{description}

\item The cosine similarity function is:

\item $$\cos (\theta)=\frac{\mathbf{A} \cdot \mathbf{B}}{\|\mathbf{A}\|\|\mathbf{B}\|}=\frac{\sum_{i=1}^{n} A_{i} B_{i}}{\sqrt{\sum_{i=1}^{n} A_{i}^{2}} \sqrt{\sum_{i=1}^{n} B_{i}^{2}}}$$

\item $A$ and $B$ represent the word vectors and $A_i$ or $B_i$ represent index i of that vector.
\item Note that if A and B are identical, you will get $cos(\theta) = 1$.
\item Otherwise, if they are the total opposite, meaning, $A= -B$, then you would get $cos(\theta) = -1$.
\item If you get $cos(\theta) =0$, that means that they are orthogonal (or perpendicular).
\item Numbers between 0 and 1 indicate a similarity score.
\item Numbers between -1-0 indicate a dissimilarity score.
\end{description}
\subsection {Euclidean distance}
\begin{description}
  \item You will now implement a function that computes the similarity between two vectors using the Euclidean distance.
Euclidean distance is defined as:

\item $$ \begin{aligned} d(\mathbf{A}, \mathbf{B})=d(\mathbf{B}, \mathbf{A}) &=\sqrt{\left(A_{1}-B_{1}\right)^{2}+\left(A_{2}-B_{2}\right)^{2}+\cdots+\left(A_{n}-B_{n}\right)^{2}} \\ &=\sqrt{\sum_{i=1}^{n}\left(A_{i}-B_{i}\right)^{2}} \end{aligned}$$

\item $n$ is the number of elements in the vector
\item $A$ and $B$ are the corresponding word vectors.
\item The more similar the words, the more likely the Euclidean distance will be close to 0.p
\end{description}

\section {Principle Component Analysis (PCA)}
\begin{description}
\item PCA is a method that projects our vectors in a space of reduced dimension, while keeping the maximum information about the original vectors in their reduced counterparts. In this case, by *maximum infomation* we mean that the Euclidean distance between the original vectors and their projected siblings is minimal. Hence vectors that were originally close in the embeddings dictionary, will produce lower dimensional vectors that are still close to each other.

\item such that similar words will be clustered next to each other. For example, the words 'sad', 'happy', 'joyful' all describe emotion and are supposed to be near each other when plotted. The words: 'oil', 'gas', and 'petroleum' all describe natural resources. Words like 'city', 'village', 'town' could be seen as synonyms and describe a

\item Before plotting the words, you need to first be able to reduce each word vector with PCA into 2 dimensions and then plot it. The steps to compute PCA are as follows:
  \begin{itemize}
  \item Mean normalize the data
  \item Compute the covariance matrix of the data ($\Sigma$).
  \item Compute the eigenvectors and the eigenvalues of your covariance matrix
  \item Multiply the first K eigenvectors by the normalized data.
\end{itemize}
\end{description}

\section {Machine Translation}
\begin{description}

\item Given dictionaries of English and French word embeddings you will create a transformation matrix `R`
\item Given an English word embedding, $\mathbf{e}$, you can multiply $\mathbf{eR}$ to get a new word embedding $\mathbf{f}$.
\item Both $\mathbf{e}$ and $\mathbf{f}$ are row vectors.
\item we can then compute the nearest neighbors to `f` in the french embeddings and recommend the word that is most similar to the transformed word embedding.

\item Find a matrix `R` that minimizes the following equation.

\item $$\arg \min _{\mathbf{R}}\| \mathbf{X R} - \mathbf{Y}\|_{F} $$

\item The Frobenius norm of a matrix $A$ (assuming it is of dimension $m,n$) is defined as the square root of the sum of the absolute squares of its elements:

\item $$\|\mathbf{A}\|_{F} \equiv \sqrt{\sum_{i=1}^{m} \sum_{j=1}^{n}\left|a_{i j}\right|^{2}}$$


\item In the real world applications, the Frobenius norm loss:

\item $$\| \mathbf{XR} - \mathbf{Y}\|_{F}$$

\item is often replaced by it's squared value divided by $m$:

\item $$ \frac{1}{m} \|  \mathbf{X R} - \mathbf{Y} \|_{F}^{2}$$

\item where $m$ is the number of examples (rows in $\mathbf{X}$).

\item The same R is found when using this loss function versus the original Frobenius norm.
\item The reason for taking the square is that it's easier to compute the gradient of the squared Frobenius.
\item The reason for dividing by $m$ is that we're more interested in the average loss per embedding than the  loss for the entire training set.
\item The loss for all training set increases with more words (training examples),
\item so taking the average helps us to track the average loss regardless of the size of the training set.
\end{description}
\subsection {Loss function L}
\begin{description}
\item The loss function will be squared Frobenoius norm of the difference between
\item matrix and its approximation, divided by the number of training examples $m$.
\item  Its formula is:
\item $$ L(X, Y, R)=\frac{1}{m}\sum_{i=1}^{m} \sum_{j=1}^{n}\left( a_{i j} \right)^{2}$$

\item where $a_{i j}$ is value in $i$th row and $j$th column of the matrix $\mathbf{XR}-\mathbf{Y}$.

\item The norm is always nonnegative (we're summing up absolute values), and so is the square.
\item When we take the square of all non-negative (positive or zero) numbers, the order of the data is preserved.
\item For example, if $3 > 2, 3^2 > 2^2$
\item Using the norm or squared norm in gradient descent results in the same location of the minimum.
\item  Squaring cancels the square root in the Frobenius norm formula. Because of the chain rule, we would have to do more calculations if we had a square root in our expression for summation.
\item Dividing the function value by the positive number doesn't change the optimum of the function, for the same reason as described above.
\item We're interested in transforming English embedding into the French. Thus, it is more important to measure average loss per embedding than the loss for the entire dictionary (which increases as the number of words in the dictionary increases).
\end{description}
\subsection {gradient descent}
\begin{description}
\item Calculate the gradient of the loss with respect to transform matrix `R`.
\item The gradient is a matrix that encodes how much a small change in `R` affect the change in the loss function.
\item The gradient gives us the direction in which we should decrease `R`
\item to minimize the loss.
\item $m$ is the number of training examples (number of rows in $X$).
\item The formula for the gradient of the loss function $L(X,Y,R)$ is:
\item $$\frac{d}{dR}L(X,Y,R)=\frac{d}{dR}\Big(\frac{1}{m}\| X R -Y\|_{F}^{2}\Big) = \frac{2}{m}X^{T} (X R - Y)$$

\subsection{fixed number of iterations}
\item You cannot rely on training loss getting low -- what you really want is the validation loss to go down, or validation accuracy to go up. And indeed - in some cases people train until validation accuracy reaches a threshold, or -- commonly known as "early stopping" -- until the validation accuracy starts to go down, which is a sign of over-fitting.
\item Why not always do "early stopping"? Well, mostly because well-regularized models on larger data-sets never stop improving. Especially in NLP, you can often continue training for months and the model will continue getting slightly and slightly better. This is also the reason why it's hard to just stop at a threshold -- unless there's an external customer setting the threshold, why stop, where do you put the threshold?
\item Stopping after a certain number of steps has the advantage that you know how long your training will take - so you can keep some sanity and not train for months. You can then try to get the best performance within this time budget. Another advantage is that you can fix your learning rate schedule -- e.g., lower the learning rate at 10% before finish, and then again more at 1% before finishing. Such learning rate schedules help a lot, but are harder to do if you don't know how long you're training.

\item  Pseudocode:
\item 1. Calculate gradient $g$ of the loss with respect to the matrix $R$.
\item 2. Update $R$ with the formula:
\item $$R_{\text{new}}= R_{\text{old}}-\alpha g$$
\item Where $\alpha$ is the learning rate, which is a scalar.

  \subsection {k-Nearest neighbors algorithm}
\item k-NN is a method which takes a vector as input and finds the other vectors in the dataset that are closest to it.
\item The 'k' is the number of "nearest neighbors" to find (e.g. k=2 finds the closest two neighbors).

  \subsection{Searching for the translation embedding}

\item Since we're approximating the translation function from English to French embeddings by a linear transformation matrix R
, most of the time we won't get the exact embedding of a French word when we transform embedding e

\item of some particular English word into the French embedding space.

\item This is where k-NN becomes really useful! By using 1-NN with eR as input, we can search for an embedding f (as a row) in the matrix Y which is the closest to the transformed vector eR.

\item  Note: Distance and similarity are pretty much opposite things.
\item We can obtain distance metric from cosine similarity, but the cosine similarity can't be used directly as the distance metric.
\item When the cosine similarity increases (towards $1$), the "distance" between the two vectors decreases (towards $0$).
\item We can define the cosine distance between $u$ and $v$ as
\item $$d_{\text{cos}}(u,v)=1-\cos(u,v)$$

  \subsection{LSH and document search}
\item In this part of the assignment, you will implement a more efficient version of k-nearest neighbors using locality sensitive hashing. You will then apply this to document search.

\item Process the tweets and represent each tweet as a vector (represent a document with a vector embedding).
\item Use locality sensitive hashing and k nearest neighbors to find tweets that are similar to a given tweet.
\item we will now implement locality sensitive hashing (LSH) to identify the most similar tweet.
\item Instead of looking at all 10,000 vectors, you can just search a subset to find its nearest neighbors
\item we can divide the vector space into regions and search within one region for nearest neighbors of a given vector.

\subsection{Bag-of-words (BOW) document models}
\item Text documents are sequences of words.
\item The ordering of words makes a difference. For example, sentences "Apple pie is
\item better than pepperoni pizza." and "Pepperoni pizza is better than apple pie"
\item have opposite meanings due to the word ordering.
\item However, for some applications, ignoring the order of words can allow
\item us to train an efficient and still effective model.
\item This approach is called Bag-of-words document model.

\item Document embedding is created by summing up the embeddings of all words
in the document.
\item If we don't know the embedding of some word, we can ignore that word.

\subsection{ Choosing the number of planes}

\item Each plane divides the space to $2$ parts.
\item So $n$ planes divide the space into $2^{n}$ hash buckets.
\item We want to organize 10,000 document vectors into buckets so that every bucket has about $~16$ vectors.
\item For that we need $\frac{10000}{16}=625$ buckets.
\item We're interested in $n$, number of planes, so that $2^{n}= 625$. Now, we can calculate $n=\log_{2}625 = 9.29 \approx 10$.

\item  In $3$-dimensional vector space, the hyperplane is a regular plane. In $2$ dimensional vector space, the hyperplane is a line.
\item Generally, the hyperplane is subspace which has dimension $1$ lower than the original vector space has.
\item A hyperplane is uniquely defined by its normal vector.
\item Normal vector $n$ of the plane $\pi$ is the vector to which all vectors in the plane $\pi$ are orthogonal (perpendicular in $3$ dimensional case).
\subsection{ Getting the hash number for a vector}

\item For each vector, we need to get a unique number associated to that vector in order to assign it to a "hash bucket".

\item Using Hyperplanes to split the vector space:
\item We can use a hyperplane to split the vector space into $2$ parts.
\item  All vectors whose dot product with a plane's normal vector is positive are on one side of the plane.
\item All vectors whose dot product with the plane's normal vector is negative are on the other side of the plane.

\item Encoding hash buckets:
\item For a vector, we can take its dot product with all the planes, then encode this information to assign the vector to a single hash bucket.
\item When the vector is pointing to the opposite side of the hyperplane than normal, encode it by 0.
\item Otherwise, if the vector is on the same side as the normal vector, encode it by 1.
\item If you calculate the dot product with each plane in the same order for every vector, you've encoded each vector's unique hash ID as a binary number, like [0, 1, 1, ... 0].

\item hash algorithm:

\item We've initialized hash table `hashes` for you. It is list of $N_{universe}$ matrices, each describes its own hash table. Each matrix has $N_{dims}$ rows and $N_{planes}$ columns. Every column of that matrix is a $N_{dims}$ dimensional normal vector for each of $N_{planes}$ hyperplanes which are used for creating buckets of the particular hash table.

\item  First multiply your vector `v`, with a corresponding plane. This will give you a vector of dimension $N_{planes}$.
\item You will then convert every element in that vector to 0 or 1.
\item You create a hash vector by doing the following: if the element is negative, it becomes a 0, otherwise you change it to a 1.
\item You then compute the unique number for the vector by iterating over $N_{planes}$
\item Then you multiply $2^i$ times the corresponding bit (0 or 1).
\item You will then store that sum in the variable $hash_{value}$.
\item $$ hash = \sum_{i=0}^{N-1} \left( 2^{i} \times h_{i} \right) $$

\end{description}
\section{Probabilistic model of pronounciation and spelling}
\begin{description}
  \subsection{auto-correction}
\item the misspelling is quite common in writing, and to transduct a word from the misspelled form to dictionary closest word, most relevant to the context for spelling, and pronunciation we utilize \textbf{Bayes Rule}, and \textbf{the noisy channel model}, and this problem can be divided into two categories:
\item 1. word error detection: in which the algorithm is run on the word in isolation.
\item 2. context error detection: where correction take place in a specific context.
\item 80\% of the misspelled words are caused by single-eror misspellings: can be divided into four categories:
  \begin{itemize}
  \item insertion: mistyping the as ther
  \item deletion: mistyping the as th
  \item substitution: mistyping the as thw
  \item transposition mistyping the as hte
  \end{itemize}
  \subsection{Bayesian inference model}
\item given a noisy word through noisy channel, \textbf{O} as our observation, we need to match it to the nearest word in the dictionary.
\item we build a vocabulary \textbf{V}, and our model ought to map noisy \textbf{O} to \textbf{$\hat{w}$}.
\item $$ \hat{w}=\underset{w \in V}{\operatorname{argmax}}P(w|O) $$
\item $$ \hat{w}=\underset{w \in V}{\operatorname{argmax}}\frac{P(O|w)P(w)}{P(O)} $$
\item since we iterate through the whole word set in the vocabulary \textbf{V} P(O) is fixed, and we can ignore it, then:
\item $$ \hat{w}=\underset{w \in V}{\operatorname{argmax}} P(O|w)P(w) $$
\item where P(w) is the \textbf{prior}, P(O|w) is the \textbf{likelihood} function.
\item p(w) is the word frequency: $\frac{frequency(w)}{size of the corpus}$, to avoid getting zero frequency we use Laplacian smoothing:
\item $$ p(w)=\frac{freq(w)+1}{N+V} $$ such that V is the size of vocabulary in this context, and N is the size of the corpus.
\item there different algorithms for error correction, and processing, among those are \textbf{minimum edit distance}, \textbf{Viterbi} \textbf{forward}, \textbf{CYK}, \textbf{Earley}.
  \subsection{Minimum edit distance}
\item It's a metric value between different noise channels for the same word, or weight for insertion, deletion, and substitution, weighting each by 1, but substitution by 2 (insertion+substitution), known as \textbf{Levenshtein} distance.
\item given two words target, and source, word distance can be calculated through Dynamic programming, laying out the target of length$\rightarrow{n}$ in the first column, and source of length $\rightarrow{m}$ in the first row, and creating matrix \textbf{distance} of size $\rightarrow{n}$ (n+1,m+1).
\item looping through each column i from 0 $\rightarrow{n}$, and each row j from 0 $\rightarrow{m}$ :
\item $$dstance[i,j] \leftarrow Min \begin{cases} \text{(distance[i-1,j] + inseration-cost($target_j$)}\\ \text{distance[i-1, j-1] + substraction-cost($source_j$,$target_i$)}\\ \text{distance[i,j-1] + insertion-cost($source_j$))}\end{cases}$$
\end{description}

\section {Grammar Weighted Automata}
\begin{description}
\item it's a weighted directed graph of finite automaton for language, to predict the probability of following word.
  \subsection{Markov chain}
    \item .
  \subsection{Hidden Markov Models HMMs}
\item it's a special time of weighted Automata, in which previous states determine the current state.
\item the weights over the directed arrows can be loaded from a given corpus as the probability of word $w_i$ followed by $w_j$, or in case of pronunciation, the probability of phone(p) $p_i$ followed by $p_j$.
\item in tagged words we can classify each word in the corpus into specific group, and build the weighted graph for the classes instead.
\item a weighted automaton is consisting of a set of states $q=<q_0q_1q_2...q_n>$, and transition states $a_{01}a_{12}a_{23}...a_{n-1}a_n$, while the input to the machine is called the Observation and denoted by $O=(o_1o_2o_3...o_t)$.
\item decoding problem: is the resolution of the underlying sequence that produce a certain observation.
\item word-detection is done through Bayesian inference $P(w|O)=P(O|w)p(w)$ if we ignore the denominator as discussed earlier.
\item this method has two important elements first the forward algorithm which is analogous to the Minimum Edit distance algorithm, yet more generalized, the latter can be seen as a special case, in the forward algorithm, the row do no just represent a sequence of characters, but does indicate the possibilities to reach to each state $q_i$ from any previous state, and instead of the calculation of the minimum, here the sum of all probabilities in current state j forward[t,j] after observing the first t observations, given the automaton $\lambda$ of paths that lead to current state of aggregated,or in other words, the likelihood of the observations times the word probability p(w) and this is calculated through multiplying three different factors: \begin{enumerate}
\item previous path probability \textit{forward[t-1,i]}.
\item transition probability $a_{ij}$.
\item observation likelihood $b_{jt}$ that the current state j matches the observation symbol t. can takes the range [0,1], 1 if there is matchs, and 0 otherwise.
\end{enumerate}
\item $$ forward[t,j] = P(o_1,o_2,...,o_t,q_t = j|\lambda)P(w) $$
\item where $q_t = j$ means the probability that the t'th state in the sequence of states is state j (current state).
\item after initializing forward[0,0]=1.
\item $$ forward[j,t] = forward[i,t-1] * a[j,i] *b[j,o_t] $$ (note that i is index of previous state),
\item the problem with Forward algorithm is that it's run through a single channel. There is more efficient graph than that of forward algorithm which enable us to track multiple of words, or sentences moving from state to another, or a more general algorithm called Viterbi.
  \subsection {Viterbi Algorithm}
  \item it's consists of three main steps, \textbf{initialization} of viterbi probability, and path matrix, and viterbi \textbf{forward}, and viterbi \textbf{backward}.
\item it's a variation, or general implementation of the forward algorithm with multiple words/sentences running simultaneously.
\item we set up a probability matrix, where each column is set for time index t, and one row for each state in the Automata graph.
\item each column has a cell for each state $q_i$.
\item the evaluation of the viterbi[t,j] is the same as in the forward algorithm.
\item there is a slight difference with the Forward algorithm, which is the viterbi maximizes the sum of all path to current state.
  \item.

    \subsection {Part of Speech tagging (POS)}
  \item starting with the phenomenon: a single word can have different meanings in different sentences.
  \item How to model such linguistic complexity in language processing?
  \item look at the following examples: \begin{itemize}
    \item The whole team played well. [adverb]
    \item You are doing well for yourself. [adjective]
    \item Well, this assignment took me forever to complete. [interjection]
    \item The well is dry. [noun]
    \item Tears were beginning to well in her eyes. [verb]
  \end{itemize}
  \item the same word is used in different contexts to mean different things with different part of speech tags.

  \item Algorithm:
    \begin{enumerate}

    \item calculate transition counts vector $C_t$ from tag $t_{i-1}$ to $t_i$ to calculate $P(t_i|t_{i-1})$.

    \item calculate emission counts vector $C_e$ of tag $t_i$ and word $w_i$ to calculate $P(w_i|t_i)$.

    \item tag counts is the count $C$ of occurrence of specific tag.

    \item create a Markov model of tag states, of transition probability from step (1), and create \textbf{transition matrix A} of all the sates as follows: $\frac{C_t(t_{i-1},t_i) + \alpha}{C(t_{i-1}) + \alpha N}$ smoothed to avoid zero, or undefined value, or in other words to calculate the probability of the relation that doesn't exis, constant $\alpha$ usually small value of 0.001, and N the number of tags.

    \item create \textbf{emission matrix B} from step(2) for the probability of observation O at specific state t:  $\frac{C_e(t_i, w_i) + \alpha}{C(t_i) + \alpha V}$. where V is the number of words in the dictionary.

    \item the formula for the Viterbi matrix is different in the first column(first word) than the rest of the columns, in the previous column the previous value term is ignored!

    \item Viterbi initialization: the previous value is set to the begining tag (i.e str --s--) let's call it start-state, of index 0, or $start_{idx}$, also to avoid getting zero value we take the logarithm of the natural e, of the transition, and emission states: $log(A[0,j]*log[j,w_i])$ = $log(A[0,j]) + log[j,w_i]$ (where j in the current tag index in the list of tags of size N, and i is the previous state index, and in the case of initialization it's 0,  $j \in N$), therefore the final initialization algorithm is:  $$V[j,0] = log(A[0,j]) + log[j,w_i]$$ $w_i$ is the current state word.

    \item Viterbi forward: $$V[j,i] = \underset{k \in N}{\operatorname{argmax}}(V[k, i-1] + log(A[k,j]) + log[j,w_i])$$

    \item Viterbi backward: starting for the last we pick the highest probability in the Viterbi matrix and record the indices of the corresponding tag, going backward up to the first column, add those tags in reverse order.
    \end{enumerate}


    \section {N-grams}
    \subsection{Smoothing}
    \begin{description}
      \item Add-k smoothing was described as a method for smoothing of the probabilities for previously unseen n-grams.
      \end{description}
    \subsection{Back-off}
    \begin{description}
    \item Back-off is a model generalization method that leverages information from lower order n-grams in case information about the high order n-grams is missing. For example, if the probability of an trigram is missing, use bigram information and so on.
    \end{description}
    \subsection{Interpolation}
    \begin{description}
    \item The other method for using probabilities of lower order n-grams is the interpolation. In this case, you use weighted probabilities of n-grams of all orders every time, not just when high order information is missing.
    \end{description}


\end{description}

\citep{xyz2}
\bibliographystyle{apalike}
\bibliography{bib0}

\begin{appendices}
  \chapter{Introduction to probabilities}
  \section{probabilities chain rule}
  \section {Naive Bayes}
  \chapter{Covariance}
  \chapter{Single Value Decomposition}
  \chapter {Exponentially weighted averages}
  \begin{description}
  \item if we have a stochastic function over vector $\Theta$, to smooths it's overshoot a bit, we can take the averages over a window of size $k$, but for large sized vector $\Theta$, it's expensive to keep all values at memory, and repeatedly run the average function over a small window, instead we can keep just the previous value through weighted averages.
  \item the weighted averages for $\Theta=(\theta^1,\theta^2,...,\theta^n)$ the weighted averaged $theta^i$: $V\theta^i=\beta V \theta^{i-1} + (1-\beta) \theta^i$.
  \item there is a bias in the weighted averages in the initial steps, we can compensate for that through \textbf{bias correction} $V\theta^i=\frac{V\theta^i}{(1-\beta^t)}$. where t is a positive integer of the first initial values, for example if we doing bias correction for the first 10 values, then $t=10$.
    \subsection {How to choose the value $\beta$ ?}
    \begin{description}
    \item it's more efficient to use average function over a window of inputs, but it's expensive in a large input set, in fact the $\beta$ value is a control of window size as it tend to fade away as the exponential gets bigger.
    \item if we expand the $dV\theta^n$: $$
      (1-\beta)\theta^{n} + \beta(1-\beta)^2\theta^{n-1} + ... + \beta(1-\beta)^{n}\theta^1
      $$ and we know that $(1-\beta)^{\frac{1}{\beta}} = \frac{1}{e} = 0.367$ then $\frac{1}{1-\beta}$ iterations this weighting term will be $\frac{1}{e}$, meaning closer to zero, and then we can interpret the weight decay algorithm as the function that take the average over the last $\frac{1}{1-\beta}$ elements.

    \end{description}
  \end{description}
  \chapter{smoothing (add-k, and add-one laplacian)}

\end{appendices}

\end{document}
